\documentclass[a4paper]{scrreprt}
 
\usepackage[german]{babel}
\usepackage[utf8]{inputenc}
\usepackage[T1]{fontenc}
\usepackage{ae}
\usepackage[bookmarks,bookmarksnumbered]{hyperref}
\usepackage{graphicx}
\usepackage{floatflt}
\usepackage{enumitem}
\usepackage{pifont}
\setlength{\parindent}{0pt}
 
\begin{document}
 
\title{Studenten}
\subtitle{ILIAS Review Plugin}
\author{SWT-Gruppe 04}
\maketitle

\tableofcontents

\chapter{Josephine Rehak - Chefprogrammiererin}
\section{Arbeit im Team}
Ich hatte zuvor bereits mit anderen Programmierern zusammengearbeitet und mich ebenso in andere Projekte einarbeiten müssen. Jedoch ist es mein erstes Gruppenprojekt mit der Zeitdauer. \\
Ein großer Vorteil der Gruppe ist es, dass sich alle Mitglieder untereinander kennen und öfters in der Universität miteinander zu tun haben. So fiel die Kooperation in der Gruppe als auch die Organisation leichter. Es war ein gutes Arbeiten in der Gruppe.

\chapter{Richard Mörbitz - Assistent}
\section{Arbeit im Team}
Auf die Arbeit im Team habe ich mit gemischten Gefühlen geblickt. Zum einen hatte ich noch nie bei so einem großen Projekt mit anderen zusammen gearbeitet und war dementsprechend gespannt, in dieser Hinsicht neue Erfahrungen zu sammeln. Andererseits wurde ich beim Einführungspraktikum RoboLab von meinen Gruppenmitgliedern hinsichtlich ihrer Fähigkeiten und Motivation stark enttäuscht.\\
Dies ist zum Glück beim Softwarepraktikum ausgeblieben. Alle Gruppenmitglieder haben ihren Teil zum Ergebnis beigetragen und ihr Handwerk verstanden, wir waren bemüht, das Praktikum als Gruppe zu bestehen. Ich saß nachmittags sehr häufig mit Max und Julius im Studentencafé ASCII. Während dieser Zeit haben wir produktiv gearbeitet, hatten aber auch unseren Spaß. Auf die gesamte Gruppe bezogen fürchte ich jedoch, dass die Aufgaben nicht immer gerecht verteilt waren. Alle diese Erlebnisse haben mir aber wichtige Erfahrungen für meine Teamfähigkeit geliefert.

\chapter{Max Friedrich - Administrator}
\section{Arbeit im Team}
Ich habe mich von Anfang an auf die Arbeit im Team gefreut, weil ich vorher meistens allein gearbeitet hatte. Zu lernen, mit anderen zusammen ein Projekt zu planen und sich auf sie zu verlassen war eine gute Erfahrung für mich. Das Arbeiten im Team bringt meiner Meinung nach viele Vorteile mit sich, weil man zum Beispiel in der Entwurfsphase viele Ideen zusammentragen und auf die Ideen der anderen aufbauen kann. 
Auch wenn es wahrscheinlich nie so ist, dass alle Mitglieder in einem Team gleich viel arbeiten, war das Arbeiten im Team eine neue und spannende Herausforderung, die auch Spaß gemacht hat. 

\chapter{Peter Merseburger - Testverantwortlicher}
\section{Arbeit im Team}
Die Arbeit im Team hat im im Großen und Ganzen gut funktioniert. Es gab jede Woche regelmäßige Treffen im ASCII, an denen ich jedoch nicht immer teilnehmen konnte. Als Kommunikationsweg hatten wir uns neben dem Mailer auch über eine Gruppe und Gruppenchat eines großen sozialen Netzwerks verbunden. Meine Teamkollegen sind sehr aufgeschlossen und hilfsbereit gewesen, Andy hat uns als Tutor super betreut und gute Tipps gegeben.  

\chapter{Julius Felchow - Sekretär}
\section{Arbeit im Team}
Einerseits ist Teamarbeit eine großartige Sache, da man einen größeren Arbeitsaufwand in weniger Zeit tätigen kann. Man muss dabei aber immer beachten, das so etwas geplant sein muss! Die Planung war in unserem Team nicht immer optimal, man kann nie garantieren, dass in einem Team alle gleichwertig viel arbeiten.\\
Ansonsten haben wir viel Teamarbeit praktiziert, wir saßen oft, meist zu dritt, Max, Richard und ich, im ASCII und haben gemeinsam am Projekt gearbeitet, dies hat immer gut funktioniert.\\
Wir haben versucht, die Aufgaben gerecht zu verteilen, weitestgehend hat dies befriedigend funktioniert.\\

\end{document}