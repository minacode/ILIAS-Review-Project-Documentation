\documentclass[a4paper]{scrreprt}
 
\usepackage[german]{babel}
\usepackage[utf8]{inputenc}
\usepackage[T1]{fontenc}
\usepackage{ae}
\usepackage[bookmarks,bookmarksnumbered]{hyperref}
\usepackage{graphicx}
\usepackage{floatflt}
\usepackage{enumitem}
\usepackage{pifont}
\setlength{\parindent}{0pt}
 
\begin{document}
 
\title{Studenten}
\subtitle{ILIAS Review Plugin}
\author{SWT-Gruppe 04}
\maketitle

\tableofcontents

\chapter{Josephine Rehak - Chefprogrammiererin}
\section{Projekt}
Am Anfang des dritten Semesters war ich sehr motiviert, da es zum ersten mal im Studium darum ging eine vollständige Anwendung zu schreiben, die vielleicht sogar in der Praxis verwendet werden würde. Das war auch mein Grund mich sofort für ein externes Projekt einzuschreiben.\\
Zum Anderen hatte ich mich sehr darauf gefreut eine Webanwendung programmieren zu dürfen, denn das mache ich seit 3 Jahren in meiner Freizeit und ist immer noch ein Hobby von mir. So war mir das Arbeiten mit PHP, XAMPP und phpmyadmin vertraut, auch wenn besonders PHP nicht zu den guten und typensicheren Programmiersprachen zählt.
\\
 Ich hatte die Hoffnung viel durch das Projekt zu lernen, zum Beispiel wie man sicher  und sauber mit PHP programmiert, welche Ordnerstrukturen nützlich sind und wie man Komplexität und Laufzeit sicher verbinden kann.
Ich wurde zu Anfang des Semesters ziemlich enttäuscht, denn besonders die Planung von Projekten durch das Aufstellen von logischen Strukturen mit Entity-Relationship- und UML-Diagrammen macht mir Spaß. Doch sollten wir keine selbstständige Software erstellen, sondern auf ein komplexes System aufbauen, dass nicht leicht zu durchschauen und oft ungenügend dokumentiert ist. So bestand das Vorwärtskommen in dem Projekt oft aus trial and error. \\
Das Projekt wurde durch das langsame Zusammenkommen der Aufgabenstellung erschwert, weswegen wir manchmal bei dem vorgeschriebenen Zeitplan ein wenig hinterher waren. So war unser Treffen mit Stakeholder Professor Wollersheim aus Leipzig erst in der 5. Woche des Projekts, am Anfang der Entwicklungsphase. \\
Die Erstellung eines eigenen und eines gemeinsamen Prototypen, sowie deren Vorstellung waren wichtige Schritte auf unserem Weg dem gewünschten Plugin näher zu kommen. Wie auch der Zeitplan, welcher der Organisation in der Gruppe einen festen Rahmen gab.  
\section{Arbeit im Team}
Ich hatte zuvor bereits mit anderen Programmierern zusammengearbeitet und mich ebenso in andere Projekte einarbeiten müssen. Jedoch ist es mein erstes Gruppenprojekt mit der Zeitdauer. \\
Ein großer Vorteil der Gruppe ist es, dass sich alle Mitglieder untereinander kennen und öfters in der Universität miteinander zu tun haben. So fiel die Kooperation in der Gruppe als auch die Organisation leichter. Es war ein gutes Arbeiten in der Gruppe.
\section{Fazit}
Jetzt zu Ende des Projekts weiß ich, dass ich programmiertechnisch nicht weit vorwärts gekommen bin, denn Kenntnisse in objektorientiertem Java und prodzeduralem PHP zu objektorientiertem PHP zu verbinden ist nur ein kleiner Schritt. Endlich eine Ahnung zu haben wie mit ILIAS umzugehen ist, wird mir nur in Verbindung mit ILIAS helfen.\\
Ich kann jedoch zugeben, dass ich bei der Dokumentation des Projektes sehr viel gelernt habe durch das Schreiben mit \LaTeX  am Pflichtenheft, der Anwenderdokumentation und das generieren der API aus den Dokumentationskommentaren im Code. Zugegeben hat selbst Dokumentation einmal ein Spaß gemacht. :)\\
Auch bei der Organisation der Gruppe habe ich als Chefprogrammierer viel dazu gelernt bei dem Absprechen mit allen Mitgliedern, Auftraggebern, Tutor und Stakeholdern, beim Setzen von Terminen und Zusammenhalten der Gruppe.
Würde ich jedoch das Projekt erneut machen, würde ich nicht den Platz des Chefprogrammierers, sondern des Administrators, wählen, da mein Assistent Richard mehr Talent beim Einarbeiten in ILIAS gezeigt hat als ich.

\chapter{Richard Mörbitz - Assistent}

\section{Projekt}

Ich hatte zum Ende des zweiten Semesters große Erwartungen an das Projekt, da ich die Lehrveranstaltung Softwaretechnologie zwar als spannend, aber doch sehr theoretisch empfunden hatte. So brannte ich darauf, die gewonnenen Erkenntnisse an einem praktischen Beispiel zu erproben. Da ich dabei auch Erfahrungen aus der realen Softwareentwicklung sammeln wollte, entschied ich mich für ein externes Praktikum.
Zu Beginn des Softwareprojekts schlugen diese Hoffnungen erst einmal ins Gegenteil um. Ich sollte eine Web-Anwendung erweitern, obwohl ich noch nie in meinem Leben in diesem Bereich gearbeitet hatte und musste mit PHP eine für mich völlig neue, nicht immer intuitive und bisweilen auch umständliche Sprache erlernen. Zudem stellte ich recht bald fest, dass ich den Arbeitsaufwand für das Projekt stark unterschätzt hatte, weil ich mir im Vorfeld keine ernsthaften Gedanken gemacht hatte, wie viel Zeit allein die Organisation in der Gruppe in Anspruch nehmen kann. Das führte schnell dazu, dass ich die empfohlene wöchentliche Arbeitszeit überschritt. Erschwerend kam hinzu, dass unsere Aufgabenstellung sehr wage war und im Zuge späterer Treffen mit Stakeholdern, die sich bis in die Hälfte der Entwurfszeit hinein erstreckten, grundlegende Änderungen erfuhr. Das machte es nicht leicht, dem Projektplan zu folgen und die Organisation in der Gruppe den Anforderungen entsprechend zu gestalten. Desweiteren gab es für uns keine Einweisung in ILIAS, sondern wir mussten die Antworten auf unsere Fragen in der offiziellen ILIAS-Dokumentation suchen, was nicht immer zufriedenstellend gelang.
Mit der Zeit wurde die Aufgabenstellung aber präziser und änderte sich nicht mehr so häufig, sodass die Analyse zum Ende kam. Während des Entwurfs konnten wir nun konkreter arbeiten und hatten durch die Prototypen handfeste Fortschritte in der Entwicklung in ILIAS. Ich steckte nun noch mehr Aufwand in das Projekt, um mich parallel in die Funktionsweise von ILIAS-Komponenten einzulesen und anhand der gewonnenen Erkenntnisse unsere Aufgabenstellung zu erfüllen. Diese Tatsache führte dazu, dass wir zu keinem feststehenden Entwurf für unser Projekt kamen, da stets noch wichtige Informationen fehlten und wir im bestehenden Betrieb testen mussten, ob unsere Ideen überhaupt in ILIAS umsetzbar sind.
Dieser Vorgang erstreckte sich bis weit in die Implementierungsphase hinein. Zu deren Ende hatten wir dann ein Produkt, dass die Grundvoraussetzungen erfüllte und nur an wenigen Stellen noch etwas Feinschliff benötigte. Daraus ergab sich für mich eine völlig neue Sicht auf das Praktikum: ich war froh, in der Vergangenheit die notwendigen Anstrengungen unternommen zu haben, weil ich nun anhand meiner neu gewonnenen Erfahrungen unser Plugin optimieren konnte.
Ich bin zwar nicht besonders stolz auf die Qualität des Softwareentwurfs, aber in Anbetracht der kurzen Zeit, die wir hatten, um uns in ein so komplexes System wie ILIAS einzulesen und während der wir unsere Pläne häufig an wechselnde Aufgabenstellungen anpassen mussten, bin ich zufrieden, dass es uns gelungen ist, ein funktionsfähiges Plugin für ILIAS zu erstellen. Ich bin froh, dass ich verstanden habe, wie Komponenten in ILIAS funktionieren und die Arbeit daran im Rahmen einer SHK fortsetzen kann. Ebenso glücklich bin ich über die neuen Erfahrungen in Softwareentwicklung und Programmierung, deren Gewinnung nur unter dem Druck dieses Praktikums möglich war.

\section{Arbeit im Team}
Auf die Arbeit im Team habe ich mit gemischten Gefühlen geblickt. Zum einen hatte ich noch nie bei so einem großen Projekt mit anderen zusammen gearbeitet und war dementsprechend gespannt, in dieser Hinsicht neue Erfahrungen zu sammeln. Andererseits wurde ich beim Einführungspraktikum RoboLab von meinen Gruppenmitgliedern hinsichtlich ihrer Fähigkeiten und Motivation stark enttäuscht.\\
Dies ist zum Glück beim Softwarepraktikum ausgeblieben. Alle Gruppenmitglieder haben ihren Teil zum Ergebnis beigetragen und ihr Handwerk verstanden, wir waren bemüht, das Praktikum als Gruppe zu bestehen. Ich saß nachmittags sehr häufig mit Max und Julius im Studentencafé ASCII. Während dieser Zeit haben wir produktiv gearbeitet, hatten aber auch unseren Spaß. Auf die gesamte Gruppe bezogen fürchte ich jedoch, dass die Aufgaben nicht immer gerecht verteilt waren. Alle diese Erlebnisse haben mir aber wichtige Erfahrungen für meine Teamfähigkeit geliefert.

\section{Fazit}
Das Praktikum war eine sehr wichtige Erfahrung, hat es mir doch gezeigt, wie man unter erheblichem Druck sowohl eine neue Programmiersprache lernen, sich in ein komplexes Online-System einarbeiten, als auch in der Gruppe ein eigenes Projekt umsetzen kann. Ich hätte vorher nicht geglaubt, dass ich dazu in der Lage bin. Allerdings hat das Praktikum stellenweise etwas zu viel Stress verursacht, sodass ich nun fürchte, andere Lernveranstaltungen vernachlässigt zu haben.

\chapter{Max Friedrich - Administrator}

\section{Projekt}

Ich habe mich auf das Projekt gefreut, weil es eine gute Abwechslung zu den Vorlesungen bilden würde. Die Programmiersprache PHP bildete die erste Hürde, weil ich mit dieser vorher nicht wirklich vertraut war. Weitere Probleme waren die schlechte Dokumentation des Systems, für das wir arbeiten mussten und MagicDraw, weil es auf einem etwas schwächeren Computer nicht läuft. Nur um Diagramme zu erstellen hätten vielleicht auch wesentlich kleinere Programme mit weniger Aufwand und Komplexität gereicht. Trotzdem ist es gut, sich mit einem Programm wie MagicDraw beschäftigt zu haben.
Ich hatte befürchtet, dass theoretische Auflagen das praktische Arbeiten erschweren würden, was sich allerdingt nicht bestätigt hatte. Vielleicht lag das auch daran, dass es sich um ein externes Praktikum handelte. 

\section{Arbeit im Team}

Ich habe mich von Anfang an auf die Arbeit im Team gefreut, weil ich vorher meistens allein gearbeitet hatte. Zu lernen, mit anderen zusammen ein Projekt zu planen und sich auf sie zu verlassen war eine gute Erfahrung für mich. Das Arbeiten im Team bringt meiner Meinung nach viele Vorteile mit sich, weil man zum Beispiel in der Entwurfsphase viele Ideen zusammentragen und auf die Ideen der anderen aufbauen kann. 
Auch wenn es wahrscheinlich nie so ist, dass alle Mitglieder in einem Team gleich viel arbeiten, war das Arbeiten im Team eine neue und spannende Herausforderung, die auch Spaß gemacht hat. 

\section{Fazit}

Das Praktikum war eine wertvolle aber auch sehr stressige Erfahrung. Mein Semester war komplett auf das Praktikum fokussiert, sodass ich die Sorge habe meine anderen Fächer nebenbei ein bisschen vernachlässigt zu haben. Trotzdem halte ich es für wichtig neben der ganzen Theorie ein richtiges Projekt bearbeiten zu müssen. Ich konnte feststellen, warum manche theoretischen Konzepte wichtig sind. Besonders Projektmanagement ist etwas, was man einfach erleben muss.

\chapter{Peter Merseburger - Testverantwortlicher}

\section{Projekt}
Nachdem ich an dem Softwaretechnologie-Modul in meinem Studium besonders viel Spaß hatte und zur Abwechslung auch mal ein gutes Prüfungsergebnis einfahren konnte, habe ich mich auch sehr auf das Praktikum gefreut. Besonders auch deswegen, da ich zu Hause nie weiß, was ich sinnvolles programmieren könnte und gerne einmal einen vollständigen Softwareentwicklungsprozess miterleben wollte. Ebenso war es mir wichtig, dass das Endprodukt nicht irgendwo in einer Schublade verschwindet, sondern produktiv eingesetzt wird. Deshalb entschied ich mich für die Möglichkeit des externen Praktikums.
Was man wohl entwickeln wird fragt man sich da. Professor Assmanns Paradebeispiel aus der Vorlesung war ein System für ein Taxiunternehmen, dessen Funktionsmerkmale ich bereits wieder vergessen habe. Als ich schließlich erfuhr, dass unser Praktikumsauftraggeber nicht ein Unternehmen aus der Wirtschaft sondern die Universität selbst ist, kam erste Skepsis auf. Aber gut, auch die Uni braucht gelegentlich ein Stück gute Software. So habe ich mich dennoch auf die Arbeit mit der ILIAS-Plattform gefreut. Letztendlich lief es darauf hinaus, dass man nicht eine Software 'from scratch' entwickelt hat, sondern sich in ein quasi fertiges System einarbeiten und dieses erweitern sollte. Leider ist der ILIAS-Quellcode relativ schlecht dokumentiert und in objektorientiertem PHP geschriebenen, was mich nicht gerade froh stimmte und sich zunehmends als problematisch herausstellte.
!!!
Rückblickend bereue ich meine eigene Entscheidung, nicht mit am Code zu arbeiten, sondern die gesamten Tests zu schreiben. Dadurch war ich mehr oder weniger White-Box-Tester und das Einlesen in den Code gestaltete sich mitunter schwierig.
\section{Arbeit im Team}
Die Arbeit im Team hat im im Großen und Ganzen gut funktioniert. Es gab jede Woche regelmäßige Treffen im ASCII, an denen ich jedoch nicht immer teilnehmen konnte. Als Kommunikationsweg hatten wir uns neben dem Mailer auch über eine Gruppe und Gruppenchat eines großen sozialen Netzwerks verbunden. Meine Teamkollegen sind sehr aufgeschlossen und hilfsbereit gewesen, Andy hat uns als Tutor super betreut und gute Tipps gegeben.
\section{Fazit}
Gut hat mir vor allen Dingen gefallen, dass ich Erfahrungen mit einer MySQL Datenbank sammeln konnte. Meine ersten Berührungen mit PHP werde ich hoffentlich schnell wieder vergessen. Vielleicht schreibe ich irgendwann mal ein Buch 'How to keep calm coding PHP'.  
Alles in allem war das Softwaretechnologie-Praktikum eine spannende Erfahrung. Ich habe einiges gelernt und wichtige Erfahrungen gesammelt.  

\chapter{Julius Felchow - Sekretär}

\section{Projekt}

Im Rahmen des Software-Technologie-Projekts habe ich einige Erfahrungen gemacht, die mich, insbesondere in Hinsicht auf Teamarbeit und Softwareentwicklung, bereichert haben. Dem System der Teamarbeit ist man zwar im Studium und natürlich auch in der Schule schon einige male begegnet, aber nicht so durchkoordiniert und strukturiert wie dieses Projekt.\\
Ich habe außerdem während des Projektes realisiert, dass ich Dinge wie Teamkoordination und Teamarbeit im allgemeinen sehr interessant finde und mir vorstellen kann, das Thema im weiterführenden Studium zu vertiefen. Das ist in jedem Fall eine sehr wichtige Erkenntnis für mich, da ich mir noch nicht viele Gedanken gemacht habe, in welche Richtung ich gehen möchte.\\
Ich habe mich mit recht großen Erwartungen in das externe Praktikum eingeschrieben und war gespannt, wie sich das entwickelt, ob man unter Umständen die Möglichkeit bekommt, Kontakte zu einer Software-Firma zu sammeln. Dem war nicht so, da die Zusammenarbeit immernoch Uni-intern war, was aber nicht weiter schlimm war. Der Vorteil war, dass es ziemlich einfach war, mit unserer Kundin Kontakt aufzunehmen, da Frau Kombrink meistens in ihrem Büro aufzufinden war. Das gefühlt einzig externe war die Arbeit mit Professor Wollersheim aus Leipzig, was ich als sehr interessant empfand. Außerdem hatte ich die Hoffnung, eine Software zu entwickeln, auf die man stolz sein kann. Stolz können wir sein, jedoch fand ich die Arbeit an der Internetbasierten Plattform nicht so befriedigend, wie eine eigene Software, aber es haben ja die meisten Gruppen internetbasierend gearbeitet.\\
Die Planung unseres Projekts war nicht immer komplett durchdacht. Es wirkte so, als wäre es eine kurzfristige Idee gewesen, die Aufgabenstellung bekamen wir recht spät und es wurden im Nachhinein oft Änderungen daran vorgenommen, was ich als belastend empfand, da wir oft unsere Diagramme und Pflichtenheft anpassen mussten. Dies war im Nachhinein aber eine lohnende Investition in Hinblick auf das Ergebnis.\\
Durch das externe Praktikum mussten wir mit PHP arbeiten, womit ich mir eine neue Programmiersprache angeeignet habe. Dies ist einerseits immer eine wichtige Erfahrung, andererseits hab ich festgestellt, dass ich PHP nicht als meine bevorzugte Programmiersprache einstufen würde.\\
Wir hatten als Team relativ häufig Probleme mit Zeitplanung, unsere Implementierungsphase begann eine Woche später als geplant. Dies lag aber sicher auch daran, dass wir unsere Analyse oft überarbeiten mussten, da sich die Anforderungen ständig änderten.\\
Analyse ist ein Punkt, an dem wir als Team lange dran saßen, da keiner sonderlich viel Elan hatte, daran zu arbeiten, aber es musste gemacht werden und ich bin (nun) der Meinung, dass es ein wichtiger Prozess ist. Wir haben in dieser Zeit viele Überlegungen getätigt, wie wir die Aufgabe umsetzen, hätten wir dies erst später gemacht (in der Implementierungsphase), wären wir über viele Probleme gestolpert, denen wir so aus dem Weg gehen konnten.\\

\section{Arbeit im Team}

Einerseits ist Teamarbeit eine großartige Sache, da man einen größeren Arbeitsaufwand in weniger Zeit tätigen kann. Man muss dabei aber immer beachten, das so etwas geplant sein muss! Die Planung war in unserem Team nicht immer optimal, man kann nie garantieren, dass in einem Team alle gleichwertig viel arbeiten.\\
Ansonsten haben wir viel Teamarbeit praktiziert, wir saßen oft, meist zu dritt, Max, Richard und ich, im ASCII und haben gemeinsam am Projekt gearbeitet, dies hat immer gut funktioniert.\\
Wir haben versucht, die Aufgaben gerecht zu verteilen, weitestgehend hat dies befriedigend funktioniert.\\

\section{Fazit}

Das Software-Projekt war eine interessante und wichtige Erfahrung für mich.\\ 
Ich habe gelernt, wie man Teamarbeit umsetzt und was für Fehler man machen kann, desweiteren eine neue Programmiersprache gelernt, wurde mit den Grundlegenden Prozessen der Softwareentwicklung vertraut gemacht und ich behersche nun LaTeX, was in meinen Augen eine große Errungenschaft ist.\\
Ich möchte einen Dank an mein Team aussprechen, besonders Max und Richard, mit denen ich immer wieder zusammen arbeiten würde.\\

\end{document}