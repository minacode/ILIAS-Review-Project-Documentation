\documentclass[a4paper]{scrreprt}
 
\usepackage[german]{babel}
\usepackage[utf8]{inputenc}
\usepackage[T1]{fontenc}
\usepackage{ae}
\usepackage[bookmarks,bookmarksnumbered]{hyperref}
\usepackage{graphicx}
\usepackage{floatflt}
\usepackage{enumitem}
\usepackage{pifont}
\setlength{\parindent}{0pt}
 
\begin{document}
 
\title{Lösung}
\subtitle{ILIAS Review Plugin}
\author{SWT-Gruppe 04}
\maketitle

\newpage

Die von uns abgegebene Lösung erfüllt alle von Frau Dr. Demuth, Frau Kombrink und Herrn Prof. Wollersheim geforderten Pflichtkriterien. Die Wunschkriterien sind nur zu einem Teil erfüllt, so wurden die Zuordnung von Fragen zu Reviewern in Form einer Checkbox-Matrix und das Speichern von Reviews zu alten Versionen einer Frage in einer History-Datenbanktabelle implementiert. Die von uns geschriebenen Klassen und Methoden sind so strukturiert, dass sie unabhängig von zukünftigen Erweiterungen funktionieren. Andererseits ist der Code offen für diese Erweiterungen, sodass neue Kundenwünsche in Zukunft leicht implementiert werden können.\\
Aufgrund der langen Einarbeitungsphase in ILIAS und der teils unzureichenden Dokumentation  konnte der Entwurf der Software nicht isoliert erfolgen, stattdessen mussten viele Ideen zunächst implementiert werden, um zu überprüfen, ob sie überhaupt in ILIAS umsetzbar wären. Daher sähe unser Produkt, würden wir jetzt mit dem während des Praktikums erworbenen Wissen eine neue Entwurfsphase starten, sicher anders aus und wäre an einigen Stellen etwas feiner.\\
Insgesamt sind wir aber sehr zufrieden, weil es uns gelungen ist, eine funktionierende und intuitiv zu bedienende Anwendung zu schreiben, die das ohnehin schon sehr umfangreiche E-Learning System ILIAS um eine gänzlich neue Funktionalität zu erweitern.



\end{document}